\documentclass[11pt]{article}
\usepackage{amsmath}
\usepackage{multicol}
\usepackage{graphicx}
\usepackage{amsmath}
\usepackage{color}
\usepackage{xcolor}


\begin{document}
\begin{flushright}
 \texttt{Computational Complexity} \hspace*{0.10cm}\textbf{$|$} \textbf{557}\hspace*{0.5cm}
\end{flushright}

\vspace*{1cm}
\large{
\textbf{12.6 Non-polynomial Time Complexity}\\
}

\vspace*{0.2cm}
Non-deterministic polynomial time is abbreviated as NP time. For these types of problems if we are able
to certify a solution anyhow then it becomes easy to prove its correctness using the polynomial time
algorithm. For this reason, NP class problems are called polynomial time verifiable. In NP problem, it
is allowed to make a guess about the answer, and then to verify the correctness of the answer done in
polynomial time. Let us consider the case of the vertex cover problem. A vertex cover of a graph G is
the set of vertices C such that each edge of G is incident to at least one vertex of C. If a set of vertices is
chosen, it can be easily checked in polynomial time whether the set of vertices is a vertex cover of G or
not. To find the minimum, such vertex cover is called minimum vertex cover problem. This problem is
in class NP.\\

\hspace*{0.5cm} NP is the set of all decision problems that can be solved in polynomial time on a non-deterministic
Turing machine. Already, we have discussed about deterministic and non-deterministic Turing
machines. The difference between a deterministic and non-deterministic Turing machine is that the
deterministic one operates like a conventional computer, performing each instruction in sequence,
forming a computational path, whereas a non-deterministic Turing machine can form a ‘branch off’
where each branch can execute a different statement in parallel, forming a computational tree. The time
required to solve the NP problem using any currently known algorithm increases very quickly as the
size of the problem grows.\\
\vspace*{0.3cm}

$Definition$: NP is defined as the set of all decision problems for which an algorithm exists which can be
carried out by a non-deterministic Turing machine in polynomial time.\\
\hspace*{0.5cm} Consider some examples.\\

\vspace*{0.3cm}
\fcolorbox{black}{blue}{\textbf{\textcolor[rgb]{1.00,1.00,1.00}{Example 12.13}}}\hspace*{0.1cm} \textbf{Prime Factorization:} A number N can be decomposed into prime numbers where
$N \geq 2$ such that $N = p _{1} \times p _{2} \times . . . . . . p _{k}$, where pi are k prime numbers. But there are
some numbers which have only two prime factors. ‘Prime factorization’ is finding which prime numbers
multiply together to make the original number.\\
\hspace*{0.5cm} Consider a number N, which has to be broken into two factors. The process is\\

\vspace*{0.2cm}
\hspace*{0.5cm} i) Guess two numbers A and B.\\
\hspace*{0.5cm} ii) Multiply B times A, and check whether the result matches with N.\\

\vspace*{0.2cm}
The checking can be performed in polynomial time.\\
\hspace*{0.5cm} The algorithm seems to be easy, but not really. In 2009, several researchers factored a 232-digit number,
utilizing several hundreds of machines. It took more than 2 years. These types of large numbers left
for prime factorization are called RSA number.
There are huge amounts of cash prizes for finding the
factors of these large numbers! Prime factorization of such large numbers is used for RSA public key
cryptography.\\

\vspace*{0.3cm}

\fcolorbox{black}{blue}{\textbf{\textcolor[rgb]{1.00,1.00,1.00}{Example 12.14}}}\hspace*{0.1cm} \textbf{Traveling Salesman Problem:} The problem is to find the minimum distance covered
by a salesman who wishes to travel around a given set of cities, travelling each
city exactly once and returning to the beginning.\\
\hspace*{0.5cm}  Let us consider that an area of travelling consists of 4 cities as given in the following. The cost (distance)
to travel from one city to another is given in the following matrix. We are assuming that the cost of
ith node to jth node is the same as the cost of jth to ith. [They may be different (if the road is one way).]

\newpage
\begin{flushleft}
    \textbf{558}\hspace*{0.1cm} \textbf{$|$} \hspace*{0.1cm} \texttt{Introduction to Automata Theory, Formal Languages and Computation}
  \end{flushleft}

\vspace*{0.5cm}
\begin{multicols}{2}
\begin{center}
\section{picture}
\includegraphics[width=4cm,height=3cm]{557.png}
\end{center}

\begin{center}
\begin{tabular}{|c|c|c|c|c|}
 \hline

 \hline
  & 1 & 2  & 3   & 4\\
\hline
1 & 0 & 20 & 35  &42\\
\hline
2 & 20& 0  & 34  &30\\
\hline
3 & 30& 34 & 0   &12\\
\hline
4 & 41& 30 &12   &0\\
 \hline

 \hline

 \hline
\end{tabular}
\end{center}
\end{multicols}


\hspace*{0.5cm} Consider a tour starts and ends on node 1. From node 1, the person has to follow the path going
to any of 2, 3, or 4 and returning to 1. In general, we can say that the tour consists of an edge
$\langle1, k\rangle , k \in V -\{1\}$ (V is the set of all vertices) and a path from k to 1. According to the condition of the
problem, the salesman can travel each city exactly once.

 Thus, the path from k to 1 must go through
all the vertices in $V -\{1, k\}$, and this path must be the shortest. Let the length of the shortest path
starting from node i and covering all the nodes exactly once and ending on node I be $l(i, S)$. If $i = 1$,
then the function becomes\\

\begin{center}

$l(1,V - \{1\}) = min \{c _{lk} +1 (k,V {1,k})\}$\\
\hspace*{0.1cm} \tiny{$2\leq k \leq n$} \\
\end{center}

$c _{1k}$ is the cost of node 1 to node k.\\
\hspace*{0.5cm} In general, it can be written as\\

\begin{center}
  $l(i,S) = min\{c _{ij} + 1 (j,S - \{j\})$ \\
  \hspace*{-0.5mm} \tiny{$j\in s$} \\
\end{center}

\vspace*{0.1cm}
where $i \neq S$.\\
\hspace*{0.5cm} If $n = 4, i = 1$, then $S = \{2, 3, 4\}$.\\
\hspace*{0.5cm} So, $l(1, \{2, 3, 4\}) = min\{[c _{12} + l(2, \{3, 4\})], [c _\{13\} + l(3, \{2, 4\})], [c _{14} + l(4, \{2, 3\})]\}$\\
\vspace*{0.2cm}

\hspace*{1.5cm} $l(2, \{3, 4\}) = min\{[c_23 + l(3, \{4\})], [c_24 + l(4, \{3\})]\}$ \\
\hspace*{1.5cm} $l(3, \{2, 4\}) = min\{[c_32 + l(2, \{4\})], [c_34 + l(4, \{2\})]\}$ \\
\hspace*{1.5cm} $l(4, \{2, 3\}) = min\{[c_42 + l(2, \{3\})], [c_43 + l(3, \{2\})]\}$ \\

\vspace*{0.2cm}

\hspace*{2cm} $l(3, \{4\}) = min\{c_34 + l(4, {\O})\} = 12 + 41 = 53 [l(4, {\O}) = c_41]$ \\
\hspace*{2cm} $l(4, \{3\}) = min\{c_43 + l(3, {\O})\} = 12 + 30 = 42 [l(3, {\O}) = c_31]$ \\
\hspace*{2cm} $l(2, \{4\}) = min\{c_24 + l(4, {\O})\} = 30 + 41 = 71 [l(4, {\O}) = c_41]$ \\
\hspace*{2cm} $l(4, \{2\}) = min\{c_42 + l(2, {\O})\} = 30 + 20 = 50 [l(2, {\O}) = c_21]$ \\
\hspace*{2cm} $l(2, \{3\}) = min\{c_23 + l(3, {\O})\} = 34 + 30 = 64 [l(3, {\O}) = c_31]$ \\
\hspace*{2cm} $l(3, \{2\}) = min\{c_32 + l(2, {\O})\} = 34 + 20 = 54 [l(2, {\O}) = c_21]$ \\
\vspace*{0.2cm}

\hspace*{1.5cm} $l(2, \{3, 4\}) = min\{[34 + 53], [30 + 42]\} = min\{87, 72\} = 72$ \\
\hspace*{1.5cm} $l(3, \{2, 4\}) = min\{[34 + 71], [12 + 50]\} = min\{105, 62\} = 62$ \\
\hspace*{1.5cm} $l(4, \{2, 3\}) = min\{[30 + 64], [12 + 54]\} = min \{94, 66\} = 66$ \\

\vspace*{0.2cm}
\hspace*{0.5cm} $l(1, \{2, 3, 4\}) = min\{[20 + 72], [35 + 62], [42 + 66]\} = min\{92, 97, 108\} = 92$ \\

\vspace*{0.3cm}
\hspace*{0.5cm} If the element of the set S is 3, there are 3 distinct choices for i. If $| S |$ is 2, there are 2 distinct choices
for i. In general, for each value of $| S |$ there are $n - 1$ distinct choices for i.\\

\newpage
\begin{flushright}
 \texttt{Computational Complexity} \hspace*{0.10cm}\textbf{$|$} \textbf{559}\hspace*{0.5cm}
\end{flushright}
\vspace*{0.5cm}

\hspace*{0.5cm} Not including 1 and I, the number of distinct sets of size k that can be constructed is $^{n - 2}C _{k}$ . The number of steps required for TSP of node n is\\

\begin{center}
\section{picture}
\includegraphics[width=11cm,height=1cm]{559.png}
\end{center}

\hspace*{0.5cm} We know that $(1+x) ^{n} = ^{n} C _{0} x ^{0} + ^{n} C _{1} x ^{1} + ^{n} C _{2} x ^{2} + ...... + ^{n} C _{n} x ^{n}$ .If we put $x = 1$, it becomes $2 ^{n}$. \\
Therefore, the simplification of the previous formula is $(n-1)2 ^{n-1}$ .\\
\hspace*{0.5cm} The complexity of the problem is $O(n2 ^{n})$, and it is an NP problem.\\
The complexity of the problem is O(n2n), and it is an NP problem.\\

\vspace*{0.3cm}
\fcolorbox{black}{blue}{\textbf{\textcolor[rgb]{1.00,1.00,1.00}{Example 12.15}}}\hspace*{0.1cm} \textbf{Vertex Cover Problem:} It is one type of optimization problem in computer science.

Given a graph G, the aim is to find the minimum vertex cover $C \subseteq G$.\\
\hspace*{0.5cm} Given a graph $G = \{V, E\}$ and an integer $k \leq V$, the problem is to find $V'$, subset of at most k vertices,
such that each edge of G is incident to at least one vertex in $V'$.\\
\hspace*{0.5cm} The algorithm can be constructed by assuming a set of k vertices C for a given graph G. We have to
check whether there exists a vertex cover of $\leq k$ vertices.\\
\hspace*{0.5cm} The checking can be done by the following steps.\\

\vspace*{0.2cm}
i) Check whether $C$ contains at most $k$ number of vertices.\\
ii) Check whether $C$ is a vertex cover of $G$.\\
\texttt{
\hspace*{2cm} $Input G = \{V, E\}, C= \{V'\}, where | V' |\leq k$\\
\hspace*{2cm} $for i = 1 to | E |$\\
\hspace*{2.5cm} $check the edges ends on V'i$\\
\hspace*{2.5cm} $remove them from \{E\}$.\\
\hspace*{2cm} $End$\\
}
\vspace*{0.2cm}
Thus, this checking operation is performed in $O(| V' |+| E |)$ i.e. in polynomial time.\\
\hspace*{0.5cm} A graph with n number of vertices can have $2 ^{n}$ number of such vertex combination. For each of the
combinations, the same checking algorithm is performed. Among these, the set of vertices with minimum
number is the minimum vertex cover of $G$. So, a minimum vertex cover algorithm is carried out
by a non-deterministic Turing machine in polynomial time.\\

\vspace*{0.5cm}
\large{
\textbf{12.7 Polynomial Time Reducibility}\\
}

\vspace*{0.2cm}
In a deterministic Turing machine for a single state and single input, there is only one move. Thus,
a problem fed to a deterministic Turing machine is either solved or not. $Polynomial-time reduction$
is a reduction which is computable by a deterministic Turing machine in polynomial time. This is
called polynomial time many-to-one reduction. A reduction which converts instances of one decision
problem into instances of a second decision problem is called many to one reduction proposed by
Emil Post (1944). Already we have learnt about Turing reduction where a problem A is reduced to a
problem B, to solve A, assuming B is already known. Many-to-one reductions are stronger than the
Turing reduction.\\
\hspace*{0.5cm} The polynomial time reducibility concept is used in the standard definitions of NP complete.\\
\hspace*{0.5cm} If $P _{1}$ and $P _{2}$ are two decision problems, then a problem $P_{1}$ is called polynomial time reducible to $P _{2}$ if
there is a polynomial time algorithm which transforms an instance of $P _{1}$ to an instance of $P _{2}$.\\


\newpage
\begin{flushleft}
    \textbf{560}\hspace*{0.1cm} \textbf{$|$} \hspace*{0.1cm} \texttt{Introduction to Automata Theory, Formal Languages and Computation}
  \end{flushleft}

\vspace*{0.5cm}
\textbf{$Theorem 1$:} If there is a polynomial time reduction from problem $P_{1}$ to problem $P_{2}$ and if problem $P_{2}$ is
in $P$, then $P_{1}$ also is in $P$.\\

\vspace*{0.2cm}
\textbf{$Proof:$} A Problem $P_{1}$ is polynomial time reducible to $P_{2}$ if there is a polynomial time algorithm which
converts an instance of $P_{1}$ to an instance of $P_{2}$. Let the size of the input of $P_{1}$ be n. According to the
condition of polynomial time reducibility, the instance of $P_{2}$ from the instance of $P_{1}$ can be obtained in
polynomial time. Let the time complexity of this conversion be $O(n^{i})$. According to the big oh notation,
f(n) is said to be O(g(n)) if there exists a positive constant C and a value k such that $f(n) \leq Cg(n)$ for all
$n \geq k$. Thus, the size of the converted instance of $P_{2}$ is less than or equal to $Cn^{i}$, where C is a constant.\\
The problem $P_{2}$ is in P. Let the complexity of solving a problem in $P_{2}$ be $O(m^{k})$ where m is the size of the
problem instance of $P_{2}$. Thus, the instance of $P_{2}$ of size $Cn^{i}$ can be solved in complexity $O((Cn^{i})^{k})$. The
total conversion time from the input in $P_{1}$ to the solution by $P_{2}$ (in other words the solution of problem
$P_{1}$) takes $O(n^{i} +(Cn^{i})^{k}) = O(n^{i} +C^{k}n^{ik}) = O(n^{ik})$. It is in polynomial time. It is proved that $P_{1}$ is in P.\\

\begin{center}
\section{picture}
\includegraphics[width=10cm,height=2cm]{560.png}
\end{center}

\large{
\textbf{12.8 Deterministic and Non-deterministic Algorithm}\\
}

\vspace*{0.2cm}
\small{An algorithm is called a deterministic algorithm if, for a function, the output generated for a given input
is the same for all trials. The behaviour of this type of algorithm is predictable as the underlying machine
follows the same sequence of states for a given input to produce output.}
 A mathematical function is deterministic. Hence, the state is known at every step of the algorithm.\\
\hspace*{0.5cm}An algorithm is called a non-deterministic algorithm if, for a function, the output generated for a
given input is different for different trails. An algorithm that solves a problem in non-deterministic polynomial
time can run in polynomial time or exponential time depending on the choices it makes during
execution. An algorithm is non-deterministic if the underlying machine follows a different sequence of
states for a given input. Some examples of non-deterministic algorithm are spanning tree construction
algorithm, searching algorithm, merge sort algorithm, vertex set covering problems, etc. To know more
about these problems, follow some books on algorithm.\\

\vspace*{0.4cm}
\large{
\textbf{12.8.1 Tractable and Intractable Problem}\\
}

\vspace*{0.2cm}

A problem is called a tractable problem if there is a polynomial time algorithm to solve it for all
instances. There exists a deterministic Turing machine to solve these problems. In other words, it can
be said that a problem is called a tractable problem, when there exists a deterministic Turing machine
M such that\\
\vspace*{0.2cm}

\begin{itemize}
  \item M runs in polynomial time on all input\\
  \item for $x \in L$, M gives output 1\\
  \item for $x \neq L$, M gives output 0.\\
\end{itemize}

\vspace*{0.2cm}
\hspace*{0.5cm} $A$ tractable problem may be of constant $[O(1)]$, logarithmic [O(log n) or $O(n \log n)$], linear
$[O(n)]$, or polynomial [$O(n^{k})$, where k is a finite number]. The upper bound of a tractable problem is
$O$(polynomial).\\


\end{document}
